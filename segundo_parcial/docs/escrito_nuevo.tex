\documentclass[11pt]{article}

\usepackage[spanish, mexico]{babel}
\usepackage{amsmath,amssymb,amsthm}
\usepackage[utf8]{inputenc}
\usepackage{multicol, caption}

\usepackage{changepage} 

\begin{document}
\author{Saúl Adrián Alvarez Tapia \and Carlos Eduardo Gil Mezta \and Jorge Antonio Rotter Vallejo \and Sergio de Jesús Arnaud Gómez}

\title{\Huge Práctica 2: Ecuaciones diferenciales ordinarias}
\date{}
\maketitle

\noindent
\textbf{\LARGE{Introducción}} 
\newline
Los problemas de valor inicial (PVI) de la forma
\begin{align*}
\dot{y} = f(t,y) \ \ \ \ y(t_0) = y_0
\end{align*}
y problemas de valores en la frontera (PVF)
\begin{align*}
\dot{y} = f(t,y) \ \ \ \ y(a) = y_a, \ \ y(b) = y_b
\end{align*}
aparecen frecuentemente en aplicaciones. Sin embargo, no siempre es posible calcular
la solución analítica, por lo que utilizar métodos numéricos que permitan aproximar
la solución en cierto intervalo es deseable. En este trabajo exploraremos algunos de
los más conocidos.
\newline

$ $\\
$ $\\ 
\noindent
\textbf{\LARGE{Ejercicios}}
\newline
%----PREGUNTA UNO -----
\textbf{1. Método de Euler}
\newline
Aplicamos el método de Euler explícito con paso $h = 0.1$ al PVI
\begin{align}
\dot{y} = 2(t+1)y, \ \ \ \ y(0) = 1
\end{align}
en el intervalo [0,1]. Observemos primero que el problema tiene solución exacta
\begin{align}
y(t) = e^{t^2+2t}
\end{align}
Y que primeros diez pasos de la solución numérica son
\begin{multicols}{3}
\noindent
$w_0$ = 1.000000\\
$w_1$ = 1.200000 \\
$w_2$ = 1.464000 \\
$w_3$ = 1.815360 \\
$w_4$ = 2.287354 \\
$w_5$ = 2.927813 \\
$w_6$ = 3.806156 \\
$w_7$ = 5.024126 \\
$w_8$ = 6.732329 \\
$w_9$ = 9.155968 \\
$w_{10}$ = 12.635236
\end{multicols}
\noindent
Por lo que el error global en $t=1$ es $|y(20)-w_{20}| = |e^3 - 12.635236| = 
7.45030106146.$ Por otro lado, el máximo de los errores locales es 0.887953046931. 
Consideremos ahora $k \in \{0, 1, \cdots, 5\}$ y para cada $k$ definamos $h_k =
0.1 \times 2^{-k}$.  Utilizando estos cinco pasos distintos, calculamos el máximo
error local del método de Euler y el orden de convergencia experimental (eoc por sus 
siglas en inglés).
\begin{table}[h]
\caption{Máximo error local variando $k$}
\centering
\begin{tabular}{cccc}
\hline \hline
$k$ & $h_k$ & error máximo & eoc \\ [0.5ex]
0 & 0.100000 & 0.5906583277433 & 0.7226540856 \\
1 & 0.050000 & 0.2439174393180 & 0.8413914232 \\
2 & 0.025000 & 0.0815839823228 & 0.9146020379 \\
3 & 0.012500 & 0.0238877045753 & 0.9556017523 \\
4 & 0.006250 & 0.0064856605817 & 0.9773510743 \\
5 & 0.003125 & 0.0016912968397 & ------------ \\
\hline
\end{tabular}
\label{tab:hresult}
\end{table}
En la tabla podemos ver que cuando el paso se reduce a la mitad, el máximo
error local decrece

$ $\\ 

%----PREGUNTA DOS -----
\noindent
\textbf{2. Trabajo, precisión y orden}
\newline
Consideremos otra vez el PVI (1), y su solución exacta, (2). En la tabla 2 comparamos
el error del método del trapecio con dos pasos distintos y de Runge-Kutta 4 (RK4). 

\begin{table}[h]
\caption{Comparación de RK4 y trapecio}
\centering
\begin{tabular}{cccc}
\hline \hline
$ $ & Trapecio($h_1$) & Trapecio($h_2$) & RK4 \\
Error global en $t=1$ & 0.7792153733  &  0.2218238872 & 0.0042700959 \\
Número de estados  &     20       &       40       &       40 \\
\hline
\end{tabular}
\end{table}
En este caso, es más conveniente cambiar a un método de mayor orden que reducir el
tamaño del paso. Tanto RK4 con paso $h_1$ como el método del trapecio con paso 
$h_2$ tienen cuarenta estados, pero el error con RK4 es significativamente menor.
Esto se debe a que en el método del trapecio, el error global es $O(h^2)$, así que al 
dividir el paso entre 2, reducimos el error por un factor de 4. En cambio, al 
cambiar a RK4, el error es $O(h^4)$, así que manteniendo  la misma h, reducimos el 
error global en un factor de (1/100)/(1/10000)=100K4 es significativamente menor.

$ $\\ 

%---- PREGUNTA TRES ----
\noindent
\textbf{3. Método del punto medio}
\newline
Trabajaremos ahora en el intervalo [0,1] con el PVI 
\begin{align*}
\dot{y} = \begin{pmatrix}
-y_1+y_2 \\
-y_1-y_2
\end{pmatrix}
\ \ \ \ y(0) = \begin{pmatrix}
0  \\ 1
\end{pmatrix}
\end{align*}
Observemos que la solución exacta es 
\begin{align*}
y(t) = \begin{pmatrix}
e^{-t}\sin t\\
e^{-t} \cos t
\end{pmatrix}
\end{align*}
Usando el método del punto medio, con $h = \frac{1}{10}$ el error global en $t=1$ es  
0.00186603257443, mientras que con $h = \frac{1}{100}$ es $1.74722645566\times 	
10^{-5}.$ Este error es consistente con la teoría, pues por ser de orden dos,
esperaríamos que al reducir el paso diez veces, el error se redujera cien (diez al 
cuadrado) veces; y dividiendo el primer error entre cien, obtenemos $1.86603257443
\times 10^{-5}$, muy cercano al experimental.

$ $\\

%---- PREGUNTA CUATRO ----
\noindent
\textbf{4. Ecuaciones \textit{stiff}}
\newline
Consideremos el PVI
\begin{align*}
\dot{y} = 5y-3y^2, \ \ y(0) = \frac{1}{2}, \ \ t \in [0,20]
\end{align*}
Observemos primero que las soluciones de equilibro de la ecuación son $y \equiv 0$ y 
$y \equiv \frac{5}{3}$.
Aplicando Euler implícito con paso constante $h = \frac{1}{2}$, obtenemos que
$y(20) \approx w_{20} = 1.\bar{6} = 5/3$, la solución de equilibrio hacia
la cual converge.
Aplicando Euler explícito con distintos pasos ($h \in \{\frac{1}{6}, \frac{1}{5}, 
\frac{1}{4}, \frac{1}{2} \}$ obtenemos convergencia a la misma solución en todas
excepto $h = \frac{1}{2}$, que aproxima $w_{20} = 2.04166028173897$.

Esto se debe a que la ecuación es \textit{stiff}, y para que la función de iteración
dada por el método de Euler explícito converja al punto fijo $w^* = \frac{5}{3}$, se
requiere $|F'(\frac{5}{3})| = |1+5h-6h\frac{5}{3}|<1$. Y esto se da si y sólo si 
$0 < h < \frac{2}{5}$. Por eso no funcionó cuando $h$ era 0.4.

Si ahora aplicamos el método de Bogacki–Shampine [Sa12]  con tolerancia $tol = 
\frac{1}{10}$  obtenemos $w_{20} = 1.6684838860728615$; y si bajamos la tolerancia 
hasta  $\frac{1}{100}$, aproximamos $w_{20} = 1.6622585157962926$. 

$ $\\ 

%---- PREGUNTA CINCO ----
\noindent
\textbf{5. Verificar orden}
\newline
Para comparar los métodos, usaremos el siguiente problema de valor inicial:
$$
\dot{y}(t) = y^2-4, \ \ y(0) = 0,  \ \ t \in [0,2]
$$

Sea $f(t,y) = y^2-4$. Dado que $\frac{\partial f}{\partial t} = 0$ y $\frac{\partial f}{\partial y} = 2y$ existen y son continuas, $f$ es de clase $C^1$ y, así, $f$ es Lipschitz en la variable $y$ en $[0,2] \times (- \infty, \infty)$. Por el teorema de existencia y unicidad, se garantiza que la solución del problema de valor inicial es única en el intervalo $[0,2]$. De hecho, por medio de separación de variables, obtenemos que la solución es:
$$
y(t) = \frac{2-2e^{4t}}{e^{4t}+1}
$$ 
A continuación se muestran los errores globales y los órdenes de convergencia experimental obtenidos con los métodos Euler explícito, Trapecio explícito, punto medio explícito y Runge-Kutta-4 con los pasos $h \in \{ 0.02, 0.01, 0.005 \}$.

\begin{table}[h]
\caption{Euler explícito}
\centering
\begin{tabular}{ccc}
\hline \hline
$h$ & error global & eoc  \\
0.02 & 0.000326129190757 & 0.938507629735  \\
0.01 & 0.000170165190836 & 0.969815078371 \\
0.005 & 8.68814971831 \times 10^{-5} & ------------  \\
\hline
\end{tabular}
\label{tab:hresult}
\end{table}

\begin{table}[h]
\caption{Trapecio explícito}
\centering
\begin{tabular}{ccc}
\hline \hline
$h$ & error global & eoc \\
0.02 & 9.87464998814 \times 10^{-6} & 2.04469956316 \\
0.01 & 2.3933477753 \times 10^{-6} & 2.02132686209 \\
0.005 & 5.89556991004 \times 10^{-7}  & ------------ \\
\hline
\end{tabular}
\label{tab:hresult}
\end{table}

\begin{table}[h]
\caption{Punto medio explícito}
\centering
\begin{tabular}{ccc}
\hline \hline
$h$ & error global & eoc \\
0.02 & 8.72843297439 \times 10^{-6} & 2.043896697077 \\
0.01 & 2.11671344053 \times 10^{-6} & 2.02104780509 \\
0.005 & 5.21514100793 \times 10^{-7}  & ------------ \\
\hline
\end{tabular}
\label{tab:hresult}
\end{table}

\begin{table}[h]
\caption{Runge-Kutta-4 }
\centering
\begin{tabular}{ccc}
\hline \hline
$h$ & error global & eoc \\
0.02 & 2.88204349275 \times 10^{-9} & 4.04473089128 \\
0.01 & 1.74628533856 \times 10^{-10} & 4.02234855171 \\
0.005 & 1.07465147892 \times 10^{-11}  & ------------ \\
\hline
\end{tabular}
\label{tab:hresult}
\end{table}

\\
\\

\newpage
De las tablas anteriores podemos ver cómo el error global se reduce aproximadamente a la mitad cuando el paso $h$ se reduce a la mitad y podemos verificar, por medio del orden experimental de convergencia, que se cumple que el método de Euler es de orden 1, que el método del trapecio y el del punto medio son de orden 2 y que el método Runge-Kutta-4 es de orden 4.

$ $\\
%---- PREGUNTA SEIS ----
\noindent
\textbf{6. Diferencias finitas}
\newline
Consideremos el problema de valores en la frontera:
$$
\ddot{y} = 10 \ \dot{y} (1-y), \ \ y(0) = 1,  \ \ y(1) = 1 - \frac{\pi}{20}
$$
que tiene como solución exacta
$$
y(t) = 1 - \frac{\pi}{20} \tan \left( \frac{\pi}{4} t \right)
$$
Usando el método de diferencias finitas de orden dos, dado que
$$
\dot{y} = \frac{y(t+h)-y(t-h)}{2h} + O(h^2)
$$
$$
\ddot{y} = \frac{y(t+h)-2y(t)+y(t-h)}{h^2} + O(h^2)
$$
Podemos escribir la discretización del problema como
$$
\frac{y(t+h)-2y(t)+y(t-h)}{h^2} = \frac{10(y(t+h)-y(t-h)) \cdot (1-y)}{2h} 
$$
Sustituyendo $w_{n-1} = y(t-h)$, $w_{n} = y(t)$ y $w_{n+1} = y(t+h)$ y reorganizando la ecuación anterior, obtenemos:
$$
w_{n+1}-2w_{n}+w_{n-1} - 5h \cdot (w_{n+1} - w_{n-1}) \cdot (1-w_{n}) = 0
$$
Definimos
$$
F(w) = 
\begin{pmatrix}
w_{2}-2w_{1}+w_{0} - 5h \cdot (w_{2} - w_{0}) \cdot (1-w_{1}) \\
w_{3}-2w_{2}+w_{1} - 5h \cdot (w_{3} - w_{1}) \cdot (1-w_{2})\\
\vdots \\
w_{i+1}-2w_{i}+w_{i-1} - 5h \cdot (w_{i+1} - w_{i-1}) \cdot (1-w_{i}) \\
\vdots \\
w_{n+1}-2w_{n}+w_{n-1} - 5h \cdot (w_{n+1} - w_{n-1}) \cdot (1-w_{n})
\end{pmatrix}
$$
Sustituyendo $w_0 = y(0) = 1 $ y $w_{n+1} = y(1) = 1 - \frac{\pi}{20}$
$$
F(w) = 
\begin{pmatrix}
w_{2}-2w_{1}+1 - 5h \cdot (w_{2} - 1) \cdot (1-w_{1}) \\
w_{3}-2w_{2}+w_{1} - 5h \cdot (w_{3} - w_{1}) \cdot (1-w_{2})\\
\vdots \\
w_{i+1}-2w_{i}+w_{i-1} - 5h \cdot (w_{i+1} - w_{i-1}) \cdot (1-w_{i}) \\
\vdots \\
1 - \frac{\pi}{20}-2w_{n}+w_{n-1} - 5h \cdot (1 - \frac{\pi}{20} - w_{n-1}) \cdot (1-w_{n})
\end{pmatrix}
$$
Para resolver $F(w) = 0$, usamos la versión matricial del método de Newton-Raphson, dada por:
$$
x_{n+1} = x_n - F(x_n) \cdot DF(x_n)^{-1}
$$
con $DF(x_n)$ el jacobiano de F.


Por eficiencia, resolvemos la siguiente ecuación de la forma $Ax = b$:
$$
DF(x_n) \cdot x = F(x_n)
$$
usando el método "solve" de Python, donde $DF(w)$ está dado por la siguiente matriz tridiagonal 
\begin{adjustwidth*}{-2.9cm}{-2.2cm}
$$
DF(w) = 
\begin{pmatrix}
-2+5h \cdot (w_2-1)  & 1-5h \cdot (1-w_1)   & 0 & \dots 0\\
1+5h \cdot (1-w_2)  & -2+5h \cdot (w_3-w_1)  & 1-5h \cdot (1-w_2) & \dots 0\\
\\
0 \ \  \ \ \ \ \ \ \ \ \ \ \ \  \ \  \ \ddots & \ \ \ \ \ \ \ \ \  \ \ \ \ \ \  \ \ \ \ \ \ \ \ \ \  \  \ddots & \ \ \ \ \ \ \ \ \ \ \ \ \  \ \ \ \  \ \ \  \ \ \ \ \ \ddots & \vdots \\
\\

0 \ \  \ \ \ \ \ \ \ \ \ \ \ \ \ \ \ \  \ \  \  & 1+5h \cdot (1-w_{n-1})  & -2+5h \cdot (w_n-w_{n-2})  & 1-5h \cdot (1-w_{n-1}) \\
0 \ \  \ \ \ \ \ \ \ \ \ \ \ \ \ \ \dots & 0 & 1+5h \cdot (1-w_{n})  & -2+5h \cdot (1 - \frac{\pi}{20}-w_{n-1})
\end{pmatrix}
$$
\end{adjustwidth*}

En la siguiente tabla se muestran los errores globales y los órdenes de convergencia experimentales del método de diferencias finitas de orden 2 para los pasos $h \in \{ 0.05, 0.025,  0.0125 \}$

\begin{table}[h]
\caption{Diferencias finitas con distintos pasos}
\centering
\begin{tabular}{ccc}
\hline \hline
$h$ & error global & eoc \\
0.05 & 0.0135543347628 & 1.06968698976 \\
0.025 & 0.00645758806692 & 1.033465018674 \\
0.0125 & 0.0031547603046  & ------------ \\
\hline
\end{tabular}
\label{tab:hresult}
\end{table}
En la tabla anterior puede verificarse cómo reducir el paso $h$ a la mitad ocasiona que el error se reduzca aproximadamente la mitad y cómo este es un método de orden 1.

\end{document}